\documentclass[10pt,]{article}
\usepackage[sc, osf]{mathpazo}
\usepackage{amssymb,amsmath}
\usepackage{ifxetex,ifluatex}
\usepackage{fixltx2e} % provides \textsubscript
\ifnum 0\ifxetex 1\fi\ifluatex 1\fi=0 % if pdftex
  \usepackage[T1]{fontenc}
  \usepackage[utf8]{inputenc}
\else % if luatex or xelatex
  \ifxetex
    \usepackage{mathspec}
  \else
    \usepackage{fontspec}
  \fi
  \defaultfontfeatures{Ligatures=TeX,Scale=MatchLowercase}
\fi
% use upquote if available, for straight quotes in verbatim environments
\IfFileExists{upquote.sty}{\usepackage{upquote}}{}
% use microtype if available
\IfFileExists{microtype.sty}{%
\usepackage{microtype}
\UseMicrotypeSet[protrusion]{basicmath} % disable protrusion for tt fonts
}{}
\usepackage[margin = .8in]{geometry}




\setlength{\emergencystretch}{3em}  % prevent overfull lines
\providecommand{\tightlist}{%
  \setlength{\itemsep}{0pt}\setlength{\parskip}{0pt}}
\setcounter{secnumdepth}{0}
% Redefines (sub)paragraphs to behave more like sections
\ifx\paragraph\undefined\else
\let\oldparagraph\paragraph
\renewcommand{\paragraph}[1]{\oldparagraph{#1}\mbox{}}
\fi
\ifx\subparagraph\undefined\else
\let\oldsubparagraph\subparagraph
\renewcommand{\subparagraph}[1]{\oldsubparagraph{#1}\mbox{}}
\fi

% Now begins the stuff that I added.
% ----------------------------------

% Custom section fonts
\usepackage{sectsty}
\sectionfont{\rmfamily\mdseries\large\bf}
\subsectionfont{\rmfamily\mdseries\normalsize\itshape}


% % Make lists without bullets
% \renewenvironment{itemize}{
%   \begin{list}{}{
%     \setlength{\leftmargin}{1.5em}
%   }
% }{
%   \end{list}
% }


% Make parskips rather than indent with lists.
\usepackage{parskip}
\usepackage{titlesec}
\titlespacing\section{0pt}{12pt plus 4pt minus 2pt}{4pt plus 2pt minus 2pt}
\titlespacing\subsection{0pt}{12pt plus 4pt minus 2pt}{4pt plus 2pt minus 2pt}

% Use fontawesome. Note: you'll need TeXLive 2015. Update.
\usepackage{fontawesome}

% Fancyhdr, as I tend to do with these personal documents.
\usepackage{fancyhdr,lastpage}
\pagestyle{fancy}
\renewcommand{\headrulewidth}{0.0pt}
\renewcommand{\footrulewidth}{0.0pt}
\lhead{}
\chead{}
\rhead{}
\lfoot{
\cfoot{\scriptsize  Christopher Junk - CV }}
\rfoot{\scriptsize \thepage/{\hypersetup{linkcolor=black}\pageref{LastPage}}}

% Always load hyperref last.
\usepackage{hyperref}
\PassOptionsToPackage{usenames,dvipsnames}{color} % color is loaded by hyperref

\hypersetup{unicode=true,
            pdftitle={Christopher Junk:  CV (Curriculum Vitae)},
            pdfauthor={Christopher Junk},
            pdfkeywords={R Markdown, academic CV, template},
            colorlinks=true,
            linkcolor=blue,
            citecolor=Blue,
            urlcolor=blue,
            breaklinks=true, bookmarks=true}
\urlstyle{same}  % don't use monospace font for urls

\begin{document}


\centerline{\huge \bf Christopher Junk}

\vspace{2 mm}

\hrule

\vspace{2 mm}

\moveleft.5\hoffset\centerline{Political Science PhD Student}
\moveleft.5\hoffset\centerline{341 Schaeffer Hall · Iowa City · IA 52242}
\moveleft.5\hoffset\centerline{ \faEnvelopeO \hspace{1 mm} \href{mailto:}{\tt \href{mailto:christopher-hollis@uiowa.edu}{\nolinkurl{christopher-hollis@uiowa.edu}}} \hspace{1 mm}  \faPhone \hspace{1 mm}  309 313 3071  \hspace{1 mm}  \faGithub \hspace{1 mm} \href{http://github.com/junkpolisci}{\tt junkpolisci} \hspace{1 mm}   \faTwitter \hspace{1 mm} \href{https:/twitter.com/Junk\_polisci}{\tt Junk\_polisci} \hspace{1 mm}    | \emph{Updated:} \today}

\vspace{2 mm}

\hrule


\section{Education}\label{education}

\emph{University of Iowa}, PhD, Political Science \hfill expected 2020

\begin{quote}
Major in statistical methodology and international relations.
\end{quote}

\emph{St.~Ambrose University}, BA, International Studies and Spanish
\hfill 2016

\section{Skills \& Expertise}\label{skills-expertise}

\textbf{Software Proficiency}: R, Stata, Python \textbf{Methodology}:
Ordinary Least Squares (OLS), Maximum Likelihood Estimation (MLE),
Markov-Chain Monte Carlo (MCMC), Time Series, Network Analysis, Spatial
Analysis\\
\textbf{Soft skills:} Collaboration, Oral/written communication, Time
management

\section{Data Experience}\label{data-experience}

\textbf{Research Assistant}, ``Informal Economies, Societal Stability
and Regime Resilience in China and Russia'' with William Reisinger,
Marina Zaloznaya, Wenfang Tang \hfill January 2019 - present

\begin{quote}
Managed and cleaned data collected through surveys in China and Russia
regards social networks and corruption.
\end{quote}

\begin{quote}
I used Stata and R to recode and restructure the dataset for use in
analyses as well as estimating and interpreting analyses and reporting
findings to the primary researchers.
\end{quote}

\textbf{Managing Editor \& Replication Analyst}, Foreign Policy Analysis
\hfill May 2018 - present

\begin{quote}
As managing editor I handle plagiarism reports, proofreading, and
general workflow of manuscripts from submission to publication.
\end{quote}

\begin{quote}
As replication analyst I attempt to replicate the empirical analyses in
manuscripts and correspond with authors to assure validity of inferences
and correct coding practices.
\end{quote}

\textbf{Collaboratory Manager}, Department of Political Science,
University of Iowa \hfill August 2018 - January 2019

\begin{quote}
I managed the updating and maintenance of the computer lab for graduate
students and undergrads enrolled in statistical courses.
\end{quote}

\begin{quote}
I served as an informal TA for undergraduate and graduate statistics
courses.
\end{quote}

\begin{quote}
Assist undergraduate and graduate students in statistical modeling and
analysis.
\end{quote}

\section{Working Papers}\label{working-papers}

\emph{Democratic Repression: Responding in Kind?}

\begin{quote}
In this paper I use webscraped events data (SPEED Project) to analyze
the dynamic relationship between public dissent and state repression.
Using a series of random effects logit models I find that democracies
typically respond to dissent with an equivalent use of force.
\end{quote}

\emph{Terrorism and Spanish Elections: Does Terrorism Impact Support for
the Central Government?}

\begin{quote}
In this paper I use Spanish polling data to assess the impact of
terrorism in Spain on support for the two major political parties. Using
time series models (ADL and GECM) I find that only particularly
catastrophic events influence public opinion, while average terrorism
has no effect.
\end{quote}

\section{Conferences Presentations}\label{conferences-presentations}

Midwest Political Science Association (2019)\\
Iowa Association of Political Scientists (2014, 2016)

\end{document}
